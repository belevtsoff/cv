%!TEX TS-program = xelatex
\documentclass[]{cv}

\begin{document}
\header{Dmytro}{Bielievtsov}
       {}


% In the aside, each new line forces a line break
\begin{aside}
  \section{contacts}
    \href{mailto:belevtsoff@gmail.com}{belevtsoff@gmail.com}
    \href{https://github.com/belevtsoff}{github.com/belevtsoff}
    +380 (93) 575 3701
  \section{languages}
    Ukrainian
    English
    German (B1)
  \section{programming}
    python, C
  \section{other}
    git, bash, latex
  \section{contributions}
    \href{https://github.com/spotify/luigi}{luigi}
    \href{https://github.com/cython/cython}{cython}
    \href{https://github.com/ibab/tensorflow-wavenet}{tensorflow-wavenet}
    \href{https://github.com/btel/SpikeSort}{SpikeSort}
    \href{https://github.com/seahboonsiew/pyspark-csv}{pyspark-csv}
  \section{pet projects}
    voice conversion
    \href{https://github.com/belevtsoff/earlPipeline}{earlPipeline}
    \href{https://github.com/belevtsoff/rdaclient.py}{rdaclient.py}
    \href{https://github.com/belevtsoff/pymsf}{pymsf}
  \section{conferences}
    DataScience Lab 2017 (Odessa, Ukraine)
    ~
    AI Ukraine 2016 (Kharkiv, Ukraine)
    ~
    Biophysical mod. 2015 (B. Honnef, Germany)
    ~
    EuroSciPy 2011 (Paris, France)
  \section{hobbies}
    rock climbing, guitar, classic cars
  \section{soft skills}
    communication
    responsibility
    teamwork
    diligence
    commitment
\end{aside}

\section{interests}

machine learning, neural networks, speech processing, computer vision, complex networks, dynamical systems, stochastic systems, differential equations, software design, distributed systems, open source movement

\section{education}

\begin{entrylist}
  \entry
    {2010-2014}
    {M.Sc. Computational Neuroscience}
    {Humboldt Universität zu Berlin}
    {Stochastic processes in the brain}
  \entry
    {2006–2010}
    {B.Sc. Computer Engineering}
    {National Aviation University, Kiev}
    {}
\end{entrylist}

\section{experience}

\begin{entrylist}
  \entry
    {since 2014}
    {IBDI, Kiev}
    {Tech lead}
    {\emph{Core responsibilities}:
        \begin{itemize}
            \item initial research, project feasibility analysis, technology selection and resource estimation.
            \item active participation in the presale and sale phases of the project's life-cycle.
            \item leading a team through the process of research and implementation of machine learning techniques for the     projects ranging from social network analysis to the speech and image processing.
        \end{itemize}
    \emph{Technologies}:
        \begin{itemize}
            \item Distributed systems: Hadoop, Spark, HDFS, Hive, Kafka etc.
            \item Python scientific stack: sklearn, scipy, numpy, matplotlib, nltk etc.
            \item Deep learning: tensorflow, keras
            \item Devops: git, docker, bash
        \end{itemize}
    }
  \entry
    {2013–2014}
    {Technische Universität Berlin}
    {Research Assistant, Programmer}
    {fMRI data analysis involving machine learning methods (multivariate pattern analysis, MVPA). Development of an open-source framework (in Python) for assembling and running fMRI data-processing pipelines on distributed systems.}
  \entry
    {2011–2013}
    {Institute for Theoretical Biology at HU Berlin}
    {Research Assistant, Programmer}
    {Electrophysiological data analysis, semiautomatic separation of spikes in multi-tetrode intracortical recordings and statistical analysis of spike-trains. Development of the open-source spike-sorting tool "SpikeSort".}
\end{entrylist}

\section{publications}
\begin{itemize}
    \item Bielievtsov D., Ladenbauer J., Obermayer K., \textit{Controlling statistical moments of stochastic dynamical networks}, Phys Rev E (2016)
    \item Bajcinca N., Hofmann S., Bielievtsov D., and Sundmacher K., \textit{Approximate ODE models for population balance systems}, Computers \& Chemical Engineering (2015)
\end{itemize}

\end{document}

